\chapter{\texorpdfstring{\centering \textbf{\MakeUppercase{Topic Specific Title}}}{Topic Specific Title}}

\section{\textbf{\MakeUppercase{the vacuum cleaner’s task}}}
\justifying
To clean an empty room, two motion trajectories can be used: spiral and zigzag.
The vacuum cleaner robot is tasked with sweeping every accessible area in the entire space while avoiding obstacles. If the robot passes by an area position, it will be cleaned; if it has not been cleaned yet, it will be.
The path planning of the coverage region (PPCR) is a task in which the robot must sweep all available free area while avoiding obstacles with an effective trajectory.
This efficient path should be short in length and contain the fewest number of turning angles and revisiting cells possible.
\section{\texttt{ The environment modeling}}
\justifying
An crucial component of the vacuum cleaner robot's duty must be ensured before it can begin, and that is the modelling of the surroundings.
The environment modeling is based on the workspace map, which is created using sensory data gathered from various sensors such as a camera, an infrared sensor, and an ultrasonic sensor.
To make the robot's movements easier, the workspace is divided into discs in the shape of a cleaning robot rather than rectangles or squares. The diagonal length of a discrete area represented by that disc cell is considered equal to the diameter of a sweeping robot.
Each disc symbolizes an impediment or a free position in the environment for the robot to inhabit. There are a maximum of eight surrounding position jth for each connection.
